\documentclass[12pt,a4paper]{scrartcl}
\usepackage[utf8]{inputenc}
\usepackage{amsmath}
\usepackage{amssymb}
\usepackage{braket}
\usepackage{amssymb}
\usepackage{graphicx}
\usepackage{xcolor}
\usepackage{float}
\usepackage[top=3cm, left=3cm, right=3cm, bottom=3cm]{geometry}
\usepackage{fancyhdr}
\usepackage[absolute]{textpos}
\numberwithin{equation}{section}
\usepackage[colorlinks=True, urlcolor=blue]{hyperref}
\pagestyle{fancy}
\fancyhead[C]{}
\fancyhead[L]{}
\fancyfoot[C]{\thepage}
\fancyhead[R]{}


\renewcommand{\exp}[1]{\mathrm{e}^{#1}}

\title{Lehmann representation of the one- and two-particle Green's function}

\begin{document}
\maketitle
\tableofcontents


\section{Lehmann representation of the Green's function}
\subsection{One-particle Green's function}
On the real axis we want to obtain the retarded Green's function
\begin{align}
 G^R_{ab}(t,t') &= -i\theta(t-t') \braket{\{ c_a(t) c^{\dagger}_b(t') \} } \\
  &= -i\theta(t-t') \Big(
   \braket{c_a(t) c^{\dagger}_b(t') } - \braket{c^{\dagger}_b(t') c_a(t)  }.
   \Big)
\end{align}
To obtain the Lehman representation we evaluate the first term and evaluate
the expectation value using the Eigenbasis $\ket{n}$ of the Hamiltonian $H$
\begin{align}
 G^>_{ab}(t,t') 
  &= -i \braket{c_a(t) c^{\dagger}_b(t') } \\
  &= \frac{-i}{Z} \sum_n \exp{-\beta E_n} \bra{n} \exp{iHt} c_a \exp{-iHt} \exp{iHt'}c^{\dagger}_b\exp{-iHt'}  \ket{n} \\
  &= \frac{-i}{Z} \sum_{n,m} \exp{-\beta E_n} \bra{n} \exp{iHt} c_a \exp{-iHt} \ket{m}\bra{m} \exp{iHt'}c^{\dagger}_b\exp{-iHt'}  \ket{n} \\
  &= \frac{-i}{Z} \sum_{n,m} \exp{-\beta E_n}\exp{i(E_n-E_m)(t-t')} \bra{n} c_a \ket{m}\bra{m} c^{\dagger}_b \ket{n}.
  \end{align}
In the same way we get for the second term
\begin{align}
 G^<_{ab}(t,t') 
  &= i \braket{c^{\dagger}_b(t') c_a(t)  } \\
  &= \frac{-i}{Z} \sum_{n,m} \exp{-\beta E_n}\exp{i(E_n-E_m)(t'-t)} \bra{n} c^{\dagger}_b \ket{m}\bra{m} c_a \ket{n} \\
  &\stackrel{n\leftrightarrow m}{=} \frac{-i}{Z} \sum_{n,m} \exp{-\beta E_m}\exp{i(E_n-E_m)(t-t')} \bra{n} c_a \ket{m}\bra{m} c^{\dagger}_b \ket{n}.
\end{align}
The full retarded Green's function is then given by
\begin{align}
 G^R_{ab}(t-t') 
&=  \theta(t-t') \frac{-i}{Z} \sum_{n,m} \left( \exp{-\beta E_n} + \exp{-\beta E_m} \right) \exp{i(E_n-E_m)(t-t')} \bra{n} c_a \ket{m}\bra{m} c^{\dagger}_b \ket{n}.
 \end{align}
A Fourier transform to frequency space yields
\begin{align}
 G_{ab}(\omega) 
 &= \int_{-\infty}^{\infty} G^R_{ab}(t) \exp{i(\omega+i\delta)t}  \, \mathrm{d}t \\
 &= \frac{-i}{Z} \sum_{n,m} \left( \exp{-\beta E_n} + \exp{-\beta E_m} \right) \bra{n} c_a \ket{m}\bra{m} c^{\dagger}_b \ket{n}
     \int_{0}^{\infty} \exp{i(w+i\delta+E_n-E_m)t}   \, \mathrm{d}t \\
&= \frac{1}{Z} \sum_{n,m} \left( \exp{-\beta E_n} + \exp{-\beta E_m} \right) 
  \frac{ \bra{n} c_a \ket{m}\bra{m} c^{\dagger}_b \ket{n} }{w+i\delta+E_n-E_m}.
\end{align}
The Matsubara Green's function is obtained by substituting $\omega+i\delta \rightarrow i\omega_n$.

We see that for a given $\ket{n}$ with $N_n$ electrons, the summation over $m$ can be restricted to the 
subspace with $N_m=N_n+1$ electrons, Or, respectively, 
for a given $\ket{m}$ with $N_m$ electrons, the summation over $n$ can be restricted to the 
subspace with $N_n=N_m-1$ electrons.


\subsection{Two-particle Green's function}
We consider the two-particle Green's function $G^{(2)}_{\uparrow \downarrow}$ as defined in the PhD Thesis of G.Rohringer
\begin{align}
G^{(2)}_{\uparrow \downarrow}(\tau_1,\tau_2,\tau_3,\tau_4)
 &= \braket{T c^{\dagger}_{\uparrow}(\tau_1) c_{\uparrow}(\tau_2) c^{\dagger}_{\downarrow}(\tau_3) c_{\downarrow}(\tau_4)  }.
\end{align}
The other spin combinations like $\uparrow \uparrow $ can by obtained from $G^{(2)}_{\uparrow \downarrow}$ by using symmetry relations.

Due to time-translational invariance we can set $\tau_4=0$, which leaves us with 
6 distinct time orderings for $\tau_1,\tau_2,\tau_3$
\begin{align}
G^{(2)}_{\uparrow \downarrow}(\tau_1,\tau_2,\tau_3,\tau_4)
&= \theta_{\tau_1 > \tau_2 > \tau_3}\braket{c^{\dagger}_{\uparrow}(\tau_1) c_{\uparrow}(\tau_2) c^{\dagger}_{\downarrow}(\tau_3) c_{\downarrow}(0)  } \nonumber\\
&- \theta_{\tau_1 > \tau_3 > \tau_2}\braket{c^{\dagger}_{\uparrow}(\tau_1) c^{\dagger}_{\downarrow}(\tau_3) c_{\uparrow}(\tau_2) c_{\downarrow}(0)  } \nonumber\\
%
&\ - \theta_{\tau_2 > \tau_1 > \tau_3}\braket{c_{\uparrow}(\tau_2) c^{\dagger}_{\uparrow}(\tau_1) c^{\dagger}_{\downarrow}(\tau_3) c_{\downarrow}(0)  } \nonumber\\
&\ + \theta_{\tau_2 > \tau_3 > \tau_1}\braket{c_{\uparrow}(\tau_2) c^{\dagger}_{\downarrow}(\tau_3) c^{\dagger}_{\uparrow}(\tau_1)  c_{\downarrow}(0)  } \nonumber\\
%
&+ \theta_{\tau_3 > \tau_1 > \tau_2 }\braket{c^{\dagger}_{\downarrow}(\tau_3) c^{\dagger}_{\uparrow}(\tau_1) c_{\uparrow}(\tau_2) c_{\downarrow}(0)  } \nonumber\\
&- \theta_{\tau_3 > \tau_2 > \tau_1 }\braket{c^{\dagger}_{\downarrow}(\tau_3) c_{\uparrow}(\tau_2) c^{\dagger}_{\uparrow}(\tau_1)  c_{\downarrow}(0)  } \nonumber\\
%
\end{align}
We now introduce the Eigenbasis $\ket{m},\ket{n},\ket{o},\ket{p}$ 
\begin{align}
 \sum_{mnop} \exp{-\beta E_m}\Big(
 &+\theta_{\tau_1 > \tau_2 > \tau_3}\braket{m|c^{\dagger}_{\uparrow}(\tau_1) |n}\braket{n| c_{\uparrow}(\tau_2)|o}\braket{o| c^{\dagger}_{\downarrow}(\tau_3) |p}\braket{p| c_{\downarrow}(0) |m} \nonumber\\
&- \theta_{\tau_1 > \tau_3 > \tau_2}\braket{m|c^{\dagger}_{\uparrow}(\tau_1)|n}\braket{n| c^{\dagger}_{\downarrow}(\tau_3) |o}\braket{o|c_{\uparrow}(\tau_2) |p}\braket{p|c_{\downarrow}(0) |m} \nonumber\\
%
&- \theta_{\tau_2 > \tau_1 > \tau_3}\braket{m|c_{\uparrow}(\tau_2)|n}\braket{n| c^{\dagger}_{\uparrow}(\tau_1) |o}\braket{o|c^{\dagger}_{\downarrow}(\tau_3) |p}\braket{p|c_{\downarrow}(0) |m} \nonumber\\
&+ \theta_{\tau_2 > \tau_3 > \tau_1}\braket{m|c_{\uparrow}(\tau_2) |n}\braket{n|c^{\dagger}_{\downarrow}(\tau_3) |o}\braket{o|c^{\dagger}_{\uparrow}(\tau_1) |p}\braket{p| c_{\downarrow}(0) |m} \nonumber\\
%
&+ \theta_{\tau_3 > \tau_1 > \tau_2 }\braket{m|c^{\dagger}_{\downarrow}(\tau_3) |n}\braket{n|c^{\dagger}_{\uparrow}(\tau_1) |o}\braket{o|c_{\uparrow}(\tau_2) |p}\braket{p|c_{\downarrow}(0) |m} \nonumber\\
&- \theta_{\tau_3 > \tau_2 > \tau_1 }\braket{m|c^{\dagger}_{\downarrow}(\tau_3) |n}\braket{n|c_{\uparrow}(\tau_2) |o}\braket{o|c^{\dagger}_{\uparrow}(\tau_1) |p}\braket{p| c_{\downarrow}(0) |m} \Big) \\
%
%
%
 =\sum_{mnop} \exp{-\beta E_m}\Big(
&+ \theta_{\tau_1 > \tau_2 > \tau_3} \exp{\tau_1(E_m-E_n)}\exp{\tau_2(E_n-E_o)}\exp{\tau_3(E_o-E_p)} \braket{m|c^{\dagger}_{\uparrow}|n}\braket{n| c_{\uparrow}|o}\braket{o| c^{\dagger}_{\downarrow} |p}\braket{p| c_{\downarrow} |m} \nonumber\\
&- \theta_{\tau_1 > \tau_3 > \tau_2} \exp{\tau_1(E_m-E_n)}\exp{\tau_3(E_n-E_o)}\exp{\tau_2(E_o-E_p)} \braket{m|c^{\dagger}_{\uparrow}|n}\braket{n| c^{\dagger}_{\downarrow} |o}\braket{o|c_{\uparrow} |p}\braket{p|c_{\downarrow} |m} \nonumber\\
%
&- \theta_{\tau_2 > \tau_1 > \tau_3} \exp{\tau_2(E_m-E_n)}\exp{\tau_1(E_n-E_o)}\exp{\tau_3(E_o-E_p)} \braket{m|c_{\uparrow}|n}\braket{n| c^{\dagger}_{\uparrow} |o}\braket{o|c^{\dagger}_{\downarrow} |p}\braket{p|c_{\downarrow} |m} \nonumber\\
&+ \theta_{\tau_2 > \tau_3 > \tau_1} \exp{\tau_2(E_m-E_n)}\exp{\tau_3(E_n-E_o)}\exp{\tau_1(E_o-E_p)} \braket{m|c_{\uparrow} |n}\braket{n|c^{\dagger}_{\downarrow} |o}\braket{o|c^{\dagger}_{\uparrow} |p}\braket{p| c_{\downarrow} |m} \nonumber\\
%
&+ \theta_{\tau_3 > \tau_1 > \tau_2} \exp{\tau_3(E_m-E_n)}\exp{\tau_1(E_n-E_o)}\exp{\tau_2(E_o-E_p)} \braket{m|c^{\dagger}_{\downarrow} |n}\braket{n|c^{\dagger}_{\uparrow} |o}\braket{o|c_{\uparrow} |p}\braket{p|c_{\downarrow} |m} \nonumber\\
&- \theta_{\tau_3 > \tau_2 > \tau_1} \exp{\tau_3(E_m-E_n)}\exp{\tau_2(E_n-E_o)}\exp{\tau_1(E_o-E_p)} \braket{m|c^{\dagger}_{\downarrow} |n}\braket{n|c_{\uparrow} |o}\braket{o|c^{\dagger}_{\uparrow} |p}\braket{p| c_{\downarrow} |m} \Big).
\end{align}
%
%
%%
%%
%%
%%
We now perform a Fourier transform $\tau_i \rightarrow i\omega_i$. The \textbf{\underline{first}} term evaluates to
\begin{align}
 &\int_0^{\beta} \mathrm{d}\tau_3 \exp{i\omega_3 \tau_3} 
 \int_0^{\beta} \mathrm{d}\tau_2 \exp{i\omega_2 \tau_2}
 \int_0^{\beta} \mathrm{d}\tau_1 \exp{i\omega_1 \tau_1}
 \Big( \exp{-\beta E_m}\theta_{\tau_1 > \tau_2 > \tau_3} \exp{\tau_1(E_m-E_n)}\exp{\tau_2(E_n-E_o)}\exp{\tau_3(E_o-E_p)} \Big) \\
 %
 &= \exp{-\beta E_m}
 \int_0^{\beta} \mathrm{d}\tau_3 \exp{\tau_3( i\omega_3 +E_o-E_p) } 
 \int_{\tau_3}^{\beta} \mathrm{d}\tau_2 \exp{\tau_2( i\omega_2 + E_n-E_o) }
 \int_{\tau_2}^{\beta} \mathrm{d}\tau_1 \exp{\tau_1( i\omega_1 + E_m-E_n) } \\
% 
 &= \exp{-\beta E_m}
 \int_0^{\beta} \mathrm{d}\tau_3 \exp{\tau_3( i\omega_3 +E_o-E_p) } 
 \int_{\tau_3}^{\beta} \mathrm{d}\tau_2 \exp{\tau_2( i\omega_2 + E_n-E_o) }
 \frac{-\exp{\beta(E_m-E_n)} - \exp{\tau_2( i\omega_1 + E_m-E_n) }}{i\omega_1 +E_m-E_n} \\
 %
 &= 
 \frac{-1}{i\omega_1 +E_m-E_n}
 \int_0^{\beta} \mathrm{d}\tau_3 \exp{\tau_3( i\omega_3 +E_o-E_p) } 
 \int_{\tau_3}^{\beta} \mathrm{d}\tau_2 \exp{\tau_2( i\omega_2 + E_n-E_o) }
 \Big( \exp{-\beta E_n} + \exp{\tau_2( i\omega_1 + E_m-E_n) }\exp{-\beta E_m} \Big) \\
 %
 &= 
 \frac{-1}{i\omega_1 +E_m-E_n}
 \int_0^{\beta} \mathrm{d}\tau_3 \exp{\tau_3( i\omega_3 +E_o-E_p) } 
 \Big(
 \exp{-\beta E_n} \frac{-\exp{\beta(E_n-E_o)} - \exp{\tau_3( i\omega_2 + E_n-E_o) }}{i\omega_2 +E_n-E_o} \nonumber \\
 & \hspace*{6cm} + \exp{-\beta E_m} \frac{+\exp{\beta(E_m-E_o)} - \exp{\tau_3(i\omega_1+ i\omega_2 + E_m-E_o) }}{i\omega_1+ i\omega_2 + E_m-E_o}
 \Big)  \\
 %
 &= 
 \frac{1}{i\omega_1 +E_m-E_n}
 \int_0^{\beta} \mathrm{d}\tau_3 \exp{\tau_3( i\omega_3 +E_o-E_p) } 
 \Big(
 \frac{\exp{-\beta E_o} + \exp{-\beta E_n}\exp{\tau_3( i\omega_2 + E_n-E_o) }}{i\omega_2 +E_n-E_o} \nonumber \\
 & \hspace*{6cm} -  \frac{\exp{-\beta E_o} - \exp{-\beta E_m}\exp{\tau_3(i\omega_1+ i\omega_2 + E_m-E_o) }}{i\omega_1+ i\omega_2 + E_m-E_o}
 \Big)  \\
 %
 &= 
 \frac{1}{i\omega_1 +E_m-E_n}
 \Big( \frac{\int_0^{\beta} \mathrm{d}\tau_3 \exp{\tau_3( i\omega_3 +E_o-E_p) } }{i\omega_2 +E_n-E_o}
 \Big[
 \exp{-\beta E_o} + \exp{-\beta E_n}\exp{\tau_3( i\omega_2 + E_n-E_o) } 
 \Big] \nonumber \\
 & \hspace*{3cm} -  \frac{\int_0^{\beta} \mathrm{d}\tau_3 \exp{\tau_3( i\omega_3 +E_o-E_p) }}{i\omega_1+ i\omega_2 + E_m-E_o}
 \Big[
 \exp{-\beta E_o} - \exp{-\beta E_m}\exp{\tau_3(i\omega_1+ i\omega_2 + E_m-E_o) }
 \Big]
 \Big)  \\
 %
  &= 
 \frac{1}{i\omega_1 +E_m-E_n}
 \Big( \frac{1}{i\omega_2 +E_n-E_o}
 \Big[
 \exp{-\beta E_o}\frac{-\exp{\beta(E_o-E_p)}-1 }{i\omega_3 +E_o-E_p} 
  + \exp{-\beta E_n}\frac{ +\exp{\beta(E_n- E_p)} -1 }{ i\omega_2 + i\omega_3 + E_n- E_p } 
 \Big] \nonumber \\
 & \hspace*{2cm} -  \frac{1}{i\omega_1+ i\omega_2 + E_m-E_o}
 \Big[
 \exp{-\beta E_o} \frac{-\exp{\beta(E_o-E_p)}-1 }{i\omega_3 +E_o-E_p} 
 - \exp{-\beta E_m} \frac{ -\exp{\beta(E_m- E_p)} -1 }{ i\omega_1 + i\omega_2 + i\omega_3 + E_m- E_p } 
 \Big]
 \Big)  \\
 %
  &= 
 \frac{1}{i\omega_1 +E_m-E_n}
 \Big( \frac{1}{i\omega_2 +E_n-E_o}
 \Big[
 \frac{-\exp{-\beta E_p}-\exp{-\beta E_o} }{i\omega_3 +E_o-E_p} 
  + \frac{ \exp{-\beta E_p} -\exp{-\beta E_n}}{ i\omega_2 + i\omega_3 + E_n- E_p } 
 \Big] \nonumber \\
 & \hspace*{2cm} -  \frac{1}{i\omega_1+ i\omega_2 + E_m-E_o}
 \Big[
  \frac{-\exp{-\beta E_p}-\exp{-\beta E_o} }{i\omega_3 +E_o-E_p} 
 -  \frac{ -\exp{-\beta E_p} -\exp{-\beta E_m} }{ i\omega_1 + i\omega_2 + i\omega_3 + E_m- E_p } 
 \Big]
 \Big)  \\
 %
   &= 
 \frac{-1}{i\omega_1 +E_m-E_n}
 \Big( \frac{1}{i\omega_2 +E_n-E_o}
 \Big[
 \frac{\exp{-\beta E_o}+\exp{-\beta E_p} }{i\omega_3 +E_o-E_p} 
  + \frac{ \exp{-\beta E_n} - \exp{-\beta E_p} }{ i\omega_2 + i\omega_3 + E_n- E_p } 
 \Big] \nonumber \\
 & \hspace*{2cm} -  \frac{1}{i\omega_1+ i\omega_2 + E_m-E_o}
 \Big[
  \frac{\exp{-\beta E_o}+\exp{-\beta E_p} }{i\omega_3 +E_o-E_p} 
 -  \frac{ \exp{-\beta E_m}+\exp{-\beta E_p} }{ i\omega_1 + i\omega_2 + i\omega_3 + E_m- E_p } 
 \Big]
 \Big)  .
 %
\end{align}
The \textbf{\underline{second}} term evaluates to
\begin{align}
 &-\int_0^{\beta} \mathrm{d}\tau_2 \exp{i\omega_2 \tau_2} 
 \int_0^{\beta} \mathrm{d}\tau_3 \exp{i\omega_3 \tau_3}
 \int_0^{\beta} \mathrm{d}\tau_1 \exp{i\omega_1 \tau_1}
 \Big( \exp{-\beta E_m}\theta_{\tau_1 > \tau_3 > \tau_2} \exp{\tau_1(E_m-E_n)}\exp{\tau_3(E_n-E_o)}\exp{\tau_2(E_o-E_p)} \Big) \\
 %
 &= -\exp{-\beta E_m}
 \int_0^{\beta} \mathrm{d}\tau_2 \exp{\tau_2( i\omega_2 +E_o-E_p)} 
 \int_0^{\beta} \mathrm{d}\tau_3 \exp{\tau_3( i\omega_3 +E_n-E_o)}
 \int_0^{\beta} \mathrm{d}\tau_1 \exp{\tau_1( i\omega_1 +E_m-E_n)},
\end{align}
which we see is precisely the same expression as the first term, except we have to swap $i\omega_2 \leftrightarrow i\omega_3$,
and flip the overall sign.
Therefore, we get $f_{mnop}(\omega_1,\omega_3,\omega_2)$, where we have defined
\begin{align}
f_{mnop}(\omega_1,\omega_2,\omega_3)
   &=
\frac{1}{i\omega_1 +E_m-E_n}
 \Big( \frac{1}{i\omega_2 +E_n-E_o}
 \Big[
 \frac{\exp{-\beta E_o}+\exp{-\beta E_p} }{i\omega_3 +E_o-E_p} 
  + \frac{ \exp{-\beta E_n} - \exp{-\beta E_p} }{ i\omega_2 + i\omega_3 + E_n- E_p } 
 \Big] \nonumber \\
 & \hspace*{2cm} -  \frac{1}{i\omega_1+ i\omega_2 + E_m-E_o}
 \Big[
  \frac{\exp{-\beta E_o}+\exp{-\beta E_p} }{i\omega_3 +E_o-E_p} 
 -  \frac{ \exp{-\beta E_m}+\exp{-\beta E_p} }{ i\omega_1 + i\omega_2 + i\omega_3 + E_m- E_p } 
 \Big]
 \Big) .
 %
\end{align}
For the \textbf{\underline{third}} term we have to swap $i\omega_1 \leftrightarrow i\omega_2$ compared to the first term and obtain $f_{mnop}(\omega_2,\omega_1,\omega_3)$

For the \textbf{\underline{fourth}} term we have to swap $i\omega_1 \leftrightarrow i\omega_3$ compared to the third term and obtain $-f_{mnop}(\omega_2,\omega_3,\omega_1)$

For the \textbf{\underline{fifth}} term we have to swap $i\omega_2 \leftrightarrow i\omega_3$ compared to the third term and obtain $-f_{mnop}(\omega_3,\omega_1,\omega_2)$

For the \textbf{\underline{sixth}} term we have to swap $i\omega_1 \leftrightarrow i\omega_2$ compared to the fifth term and obtain $f_{mnop}(\omega_3,\omega_2,\omega_1)$

Collecting all terms we obtain the final expression for the two-particle Green's function
\begin{align}
 G^{(2)}_{\uparrow \downarrow}(\omega_1,\omega_2,\omega_3)
=  \sum_{mnop} \Big( 
 % 1st%%%%%%%%%%%%%%%%%%%%%%%%%%%%%%%%%%%%%%%%%%%%%%%%%%
    & - \braket{m|c^{\dagger}_{\uparrow}|n}\braket{n| c_{\uparrow}|o}\braket{o| c^{\dagger}_{\downarrow} |p}\braket{p| c_{\downarrow} |m}
        f_{mnop}(\omega_1,\omega_2,\omega_3)\nonumber\\
 %2nd%%%%%%%%%%%%%%%%%%%%%%%%%%%%%%%%%%%%%%%%%%%%%%%%%%
    &+ \braket{m|c^{\dagger}_{\uparrow}|n}\braket{n| c^{\dagger}_{\downarrow} |o}\braket{o|c_{\uparrow} |p}\braket{p|c_{\downarrow} |m}
       f_{mnop}(\omega_1,\omega_3,\omega_2) \nonumber  \\
 %3rd%%%%%%%%%%%%%%%%%%%%%%%%%%%%%%%%%%%%%%%%%%%%%%%%%%
     &+\braket{m|c_{\uparrow}|n}\braket{n| c^{\dagger}_{\uparrow} |o}\braket{o|c^{\dagger}_{\downarrow} |p}\braket{p|c_{\downarrow} |m}
       f_{mnop}(\omega_2,\omega_1,\omega_3) \nonumber\\
 %4th%%%%%%%%%%%%%%%%%%%%%%%%%%%%%%%%%%%%%%%%%%%%%%%%%%
     &-\braket{m|c_{\uparrow} |n}\braket{n|c^{\dagger}_{\downarrow} |o}\braket{o|c^{\dagger}_{\uparrow} |p}\braket{p| c_{\downarrow} |m}
       f_{mnop}(\omega_2,\omega_3,\omega_1)\nonumber\\
 %5th%%%%%%%%%%%%%%%%%%%%%%%%%%%%%%%%%%%%%%%%%%%%%%%%%%
     &-\braket{m|c^{\dagger}_{\downarrow} |n}\braket{n|c^{\dagger}_{\uparrow} |o}\braket{o|c_{\uparrow} |p}\braket{p|c_{\downarrow} |m}
       f_{mnop}(\omega_3,\omega_1,\omega_2) \nonumber\\
 %6th%%%%%%%%%%%%%%%%%%%%%%%%%%%%%%%%%%%%%%%%%%%%%%%%%%
     &+\braket{m|c^{\dagger}_{\downarrow} |n}\braket{n|c_{\uparrow} |o}\braket{o|c^{\dagger}_{\uparrow} |p}\braket{p| c_{\downarrow} |m}
       f_{mnop}(\omega_3,\omega_2,\omega_1) \Big.
\end{align}
When analyzing the overlap elements we see that each term corresponds to the following transitions: Starting with $\ket{m}$
being a state $m(N,S)$ with electron number $N$ and total spin $S$ we get  by going from right to left
\begin{align}
 (1): \ \  & m(N,S) \rightarrow p(N-1,S+1) \rightarrow o(N,S)    &\rightarrow n(N-1,S-1)  \rightarrow m(N,S) \nonumber \\
 (2): \ \  & m(N,S) \rightarrow p(N-1,S+1) \rightarrow o(N-2,S)  &\rightarrow n(N-1,S-1)  \rightarrow m(N,S) \nonumber\\
 (3): \ \  & m(N,S) \rightarrow p(N-1,S+1) \rightarrow o(N,S)    &\rightarrow n(N+1,S+1)  \rightarrow m(N,S) \nonumber\\
 (4): \ \  & m(N,S) \rightarrow p(N-1,S+1) \rightarrow o(N,S+2)  &\rightarrow n(N+1,S+1)  \rightarrow m(N,S) \nonumber\\
 (5): \ \  & m(N,S) \rightarrow p(N-1,S+1) \rightarrow o(N-2,S)  &\rightarrow n(N-1,S+1)  \rightarrow m(N,S) \nonumber\\
 (6): \ \  & m(N,S) \rightarrow p(N-1,S+1) \rightarrow o(N,S+2)  &\rightarrow n(N-1,S+1)  \rightarrow m(N,S) 
\end{align}
So we rearrange the terms grouping them by similar transitions in order to reuse results as much as possible in the code
\begin{align}
 (1): \ \  & m(N,S) \rightarrow p(N-1,S+1) \rightarrow o(N,S)    &\rightarrow n(N-1,S-1)  \rightarrow m(N,S) \nonumber \\
 (3): \ \  & m(N,S) \rightarrow p(N-1,S+1) \rightarrow o(N,S)    &\rightarrow n(N+1,S+1)  \rightarrow m(N,S) \nonumber\\
 (4): \ \  & m(N,S) \rightarrow p(N-1,S+1) \rightarrow o(N,S+2)  &\rightarrow n(N+1,S+1)  \rightarrow m(N,S) \nonumber\\
 (6): \ \  & m(N,S) \rightarrow p(N-1,S+1) \rightarrow o(N,S+2)  &\rightarrow n(N-1,S+1)  \rightarrow m(N,S) \nonumber\\
 (2): \ \  & m(N,S) \rightarrow p(N-1,S+1) \rightarrow o(N-2,S)  &\rightarrow n(N-1,S-1)  \rightarrow m(N,S) \nonumber\\
 (5): \ \  & m(N,S) \rightarrow p(N-1,S+1) \rightarrow o(N-2,S)  &\rightarrow n(N-1,S+1)  \rightarrow m(N,S) 
\end{align}
which results in 
\begin{align}
 G^{(2)}_{\uparrow \downarrow}(\omega_1,\omega_2,\omega_3)
=  \sum_{mnop} \Big( 
 % 1st%%%%%%%%%%%%%%%%%%%%%%%%%%%%%%%%%%%%%%%%%%%%%%%%%%
    & - \braket{m|c^{\dagger}_{\uparrow}|n}\braket{n| c_{\uparrow}|o} {\color{red}\braket{o| c^{\dagger}_{\downarrow} |p}\braket{p| c_{\downarrow} |m} }
        f_{mnop}(\omega_1,\omega_2,\omega_3)\nonumber\\
 %3rd%%%%%%%%%%%%%%%%%%%%%%%%%%%%%%%%%%%%%%%%%%%%%%%%%%
     &+\braket{m|c_{\uparrow}|n}\braket{n| c^{\dagger}_{\uparrow} |o} {\color{red}\braket{o|c^{\dagger}_{\downarrow} |p}\braket{p|c_{\downarrow} |m} }
       f_{mnop}(\omega_2,\omega_1,\omega_3) \nonumber\\
 %4th%%%%%%%%%%%%%%%%%%%%%%%%%%%%%%%%%%%%%%%%%%%%%%%%%%
     &-\braket{m|c_{\uparrow} |n}\braket{n|c^{\dagger}_{\downarrow}|o} {\color{olive}\braket{o|c^{\dagger}_{\uparrow} |p}\braket{p| c_{\downarrow} |m} }
       f_{mnop}(\omega_2,\omega_3,\omega_1)\nonumber\\
 %6th%%%%%%%%%%%%%%%%%%%%%%%%%%%%%%%%%%%%%%%%%%%%%%%%%%
     &+\braket{m|c^{\dagger}_{\downarrow} |n}\braket{n|c_{\uparrow} |o} {\color{olive}\braket{o|c^{\dagger}_{\uparrow} |p}\braket{p| c_{\downarrow} |m} }
       f_{mnop}(\omega_3,\omega_2,\omega_1) \nonumber \\
 %2nd%%%%%%%%%%%%%%%%%%%%%%%%%%%%%%%%%%%%%%%%%%%%%%%%%%
    &+ \braket{m|c^{\dagger}_{\uparrow}|n}\braket{n| c^{\dagger}_{\downarrow} |o} {\color{blue}\braket{o|c_{\uparrow} |p}\braket{p|c_{\downarrow} |m} }
       f_{mnop}(\omega_1,\omega_3,\omega_2) \nonumber  \\
 %5th%%%%%%%%%%%%%%%%%%%%%%%%%%%%%%%%%%%%%%%%%%%%%%%%%%
     &-\braket{m|c^{\dagger}_{\downarrow} |n}\braket{n|c^{\dagger}_{\uparrow} |o} {\color{blue}\braket{o|c_{\uparrow} |p}\braket{p|c_{\downarrow} |m} }
       f_{mnop}(\omega_3,\omega_1,\omega_2) \Big) \\
%       
%       
   =  \sum_{mnop} \Big[ 
 % 1st+3rd%%%%%%%%%%%%%%%%%%%%%%%%%%%%%%%%%%%%%%%%%%%%%%%%%%
     \big( - \braket{m|c^{\dagger}_{\uparrow}|n}\braket{n| c_{\uparrow}|o} & f_{mnop}(\omega_1,\omega_2,\omega_3)
           + \braket{m|c_{\uparrow}|n}\braket{n| c^{\dagger}_{\uparrow} |o}  f_{mnop}(\omega_2,\omega_1,\omega_3) \big) 
            {\color{red}\braket{o|c^{\dagger}_{\downarrow} |p}\braket{p|c_{\downarrow} |m} }         \nonumber\\
 %4th+6th%%%%%%%%%%%%%%%%%%%%%%%%%%%%%%%%%%%%%%%%%%%%%%%%%%
     \big(-\braket{m|c_{\uparrow} |n}\braket{n|c^{\dagger}_{\downarrow}|o} & f_{mnop}(\omega_2,\omega_3,\omega_1)
          +\braket{m|c^{\dagger}_{\downarrow} |n}\braket{n|c_{\uparrow} |o}  f_{mnop}(\omega_3,\omega_2,\omega_1) \big)
      {\color{olive}\braket{o|c^{\dagger}_{\uparrow} |p}\braket{p| c_{\downarrow} |m} }   \nonumber \\
 %2nd+5th%%%%%%%%%%%%%%%%%%%%%%%%%%%%%%%%%%%%%%%%%%%%%%%%%%
     \big(+\braket{m|c^{\dagger}_{\uparrow}|n}\braket{n| c^{\dagger}_{\downarrow} |o} & f_{mnop}(\omega_1,\omega_3,\omega_2)
         -\braket{m|c^{\dagger}_{\downarrow} |n}\braket{n|c^{\dagger}_{\uparrow} |o}    f_{mnop}(\omega_3,\omega_1,\omega_2) \big) 
         {\color{blue}\braket{o|c_{\uparrow} |p}\braket{p|c_{\downarrow} |m} }   \Big]
\end{align}
%
%%
%%
\end{document}
